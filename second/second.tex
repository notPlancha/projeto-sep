\documentclass[12pt]{../diazessay}

% packages {
  \usepackage[backend=biber, style=apa]{biblatex}
  \usepackage{csquotes} % for apa style
  \usepackage[utf8]{inputenc}
  \usepackage{times} % Times new Roman
  \usepackage{titlesec} % change section fount
  \usepackage{xcolor} % colors
% }

\titleformat*{\section}{\normalfont\small\color{lightgray}}
\titlespacing{\section}{0pt}{15pt}{2pt}

\titleformat*{\subsection}{\normalfont\normalsize\bfseries}
\titlespacing{\subsection}{5pt}{-2pt}{3pt}

\title{\textbf{TODO Título} \\ {\Large\itshape TODO subtítulo}}
\author{\textbf{Plancha; 105289} \\ \textit{ISCTE-IUL}}
\date{\today , Versão 0.0.1}

\addbibresource{referLink.bib}

\begin{document}
\maketitle

\section*{ClearviewAI}
Em 2020, Matthias Marx apresentou uma queixa dentro do Regulamento Geral sobre a Proteção de Dados (RGPD) contra Clearview AI, uma empresa americana expecializada em reconhecimento facial (RF), por ter guardado e processado as suas fotos públicas sem o seu conhecimento e consentimento \parencite{wired}. Embora pareça intuitívo que Marx esteja no seu direito da reclamação, devido à quebra de vários artigos do RGPD, incluindo artigos 13 e 14, questões morais sobre o caso e o próprio processo de recolha e uso dos dados executado pela empresa podem ser levantadas.

Neste ensaio, estas questões vão ser analisadas usando o método de Bynum \parencite{Bynum}, de forma a entender melhor o caso e a sua relevância, bem como a sua importância para o futuro da privacidade e proteção de dados, em termos de ética digital e de responsabilidade social.

\section*{Análise do caso}
\subsection*{Ponto de vista ético}
Este caso possivelmente envolve questões éticas em vários valores éticos, incluindo potencialmente a privacidade, segurança, propriedade intelectual e consentimento de Marx, a liberdade de expressão e informação da companhia, e da privacidade e segurança pública e privada dos cidadãos.

\subsubsection*{Participantes}
O caso em questão envolve os seguintes participantes\parencite{first}: 
\begin{itemize}
  \item[Clearview AI:] A empresa criar a sua ferramenta de RF, criando perfis biométricos de pessoas a partir das suas fotos públicadas em redes sociais, blogs, ou qualquer outro site que a ferramenta tenha acesso a, de forma a combater crime, sem o consentimento e conhecimento dos indivíduos;
  \item[Matthias Marx:] O indivíduo que apresentou a queixa, que sentiu que a sua privacidade tenha sido quebrada após um Pedido de Acesso dos Dados do Titular (PADT) à empresa ter revelado as suas fotos associados ao seu nome, apenas com reconhecimento da sua cara tenha sido feita com a sua autorização; Marx também não garantiu que as suas fotos não tenham sido públicadas publicamente por terceiros ou por ele mesmo, tornando tais fotografias acessiveis a qualquer pessoa (ou máquina) com acesso à internet;
  \item[Agentes não humanos: ] As ferramentas que levou à queixa foram o \textit{web crawler}, a base de dados e o sistema de RF. De forma a facilitar a descrição, cada um deles vai ter o nome de \textsc{a\_wc}, \textsc{a\_bd} e \textsc{a\_rf}, respectivamente.
  \item[Engenheiros do sistema] Os engenheiros que criaram a ferramenta de RF e usaram técnicas de \textit{web crawling} para recolher e guardar as fotos públicas de indivíduos, sem o seu consentimento, podem ter potencialmente ter quebrado código de conduta e ética profissional, na construção do programa;
  \item[Reguladora de Alemanha:] A autoridade reguladora alemã processou a queixa de Marx sobre a quebra do RGPD.
\end{itemize}

\subsubsection*{Questões éticas e problemas}
\textsc{a\_wc} guardou as imagens de Marx sem o seu consentimento, de forma a serem identificadas pela \textsc{a\_rf}. Quem é o responsável aqui? A quebra de privacidade e do RGPD de indivíduos da União Europeia foi intencional ou uma consequência não prevista? Foram essas quebras necessárias para a segurança pública? As quebras foram feitas pelo processo de qual ferramenta/combinação de ferramentas: \textsc{a\_wc}, \textsc{a\_bd} ou \textsc{a\_rf}? Quem é responsável humano por estas quebras; o CEO, os engenheiros ou Marx? Se a quebra de privacidade fosse não intencional, quem é/são o/os responsável/eis? Foi apenas um acidente ou uma falha de segurança? A empresa está a recolher fotos de indivíduos fora dos Estados Unidos. Há alguma forma de de impedir este resultado? Deve a companhia importar-se com esses indivíduos, sendo que esses não são o foco da empresa, e se a ferramenta apenas ser usada no país, deve ela preocupar-se com a identificação de indivíduos fora dele? Se a ferramenta for suficientemente correta no seu reconhecimento e nos seus perfis, há possibilidade dos dados de Marx e outros serem expostos e usados de forma imoral ou ilegal? A própria ferramenta seria ilegal se fosse usada na Europa? Se os agentes fossem roubados por terceiros, seria possível que a segurança e privacidade de Marx estivesse em risco? Este caso podia ter sido evitado de alguma forma?
\subsubsection*{Análise sistémica}

\printbibliography[title=Referências]
\end{document}
