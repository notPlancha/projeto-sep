\documentclass[12pt]{../diazessay}

% packages {
  \usepackage[backend=biber, style=apa]{biblatex}
  \usepackage{csquotes} % for apa style
  \usepackage[utf8]{inputenc}
  \usepackage{times} % Times new Roman
  \usepackage{sectsty} % change section fount
  \usepackage{xcolor} % colors
% }

\sectionfont{\normalfont\footnotesize\textcolor{lightgray!70}}


\title{\textbf{TODO Título} \\ {\Large\itshape TODO subtítulo}}
\author{\textbf{Plancha; 105289} \\ \textit{ISCTE-IUL}}
\date{\today , Versão 0.0.1}

\addbibresource{referLink.bib}

\begin{document}
\maketitle

\section*{ClearviewAI}
Em 2020, Matthias Marx apresentou uma queixa dentro do Regulamento Geral sobre a Proteção de Dados (RGPD) contra Clearview AI, uma empresa americana expecializada em reconhecimento facial, por ter guardado e processado as suas fotos públicas sem o seu conhecimento e consentimento \parencite{wired}.


\printbibliography[title=Referências]
\end{document}
