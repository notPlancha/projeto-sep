\documentclass[portuguese, 12pt]{../diazessay}

% packages {
  \usepackage[backend=biber, style=apa]{biblatex}
  \usepackage{csquotes} % for apa style
  \usepackage[utf8]{inputenc}
  \usepackage{times} % Times new Roman
  \usepackage{textcomp} % for º
% }

\title{\textbf{\textit{\textit{Clearview AI}} e Matthias Marx} \\ {\Large\itshape Trabalho elaborado como parte 1 no âmbito da Unidade Curricular de Segurança, Ética e Privacidade do 2\textdegree ano da Licenciatura em Ciência de Dados}}
\author{\textbf{Plancha; 105289} \\ \textit{ISCTE-IUL}}
\date{\today , Versão 1.0.1}



\addbibresource{referLink.bib}

\begin{document}
\maketitle
\textit{Clearview AI} é uma empresa americana privada sediada nos Estados Unidos, conhecida pela sua ferramenta controversista \parencite{nytClearview, CVBan} de reconhecimento facial, desenvolvida para "investigar crimes, melhorar a segurança pública e provisionar justiça para vítimas" \parencite{overview}. Esta ferramenta coleta de forma automática fotos e dados de websites e redes sociais públicos para criar perfis biométricos de qualquer pessoa, sem consentimento delas \parencite{EUpresp}. Em 2020, \textit{Clearview AI} continha mais de 3 biliões de imagens publicadas na Internet \parencite{EUpresp}.

Em 2020, depois de uma investigação da jornalista Kashmir Hill, Matthias Marx, um residente de Hamburgo e membro de \textit{Chaos Cumputer Club} \parencite{LegalComp}, apresentou uma reclamação à sua reguladora local de que as suas fotos foram usadas para gerar um perfil biométrico sem o seu conhecimento, quebrando o Regulamento Geral sobre a Proteção de Dados (RGPD), após um Pedido de Acesso dos Dados do Titular mostrar tais fotos guardadas. Ainda hoje não está claro se o caso foi resolvido ou não \parencite{wired}. Marx acredita que não é possível que Clearview consiga apagar uma cara da sua base de dados diretamente, pois a tecnologia está constantemente a monitorizar a Internet, e a ferramenta "simplesmente iria catalogá-lo novamente" \parencite{wired}.

Esta foi a primeira queixa de RGPD feita sobre a empresa, e muitas outras sucederam \parencite{LegalComp}. A empresa planeia ter cem mil milhões de imagens em meio ano \parencite{expansion}.


\printbibliography[title=Referências]
\end{document}
